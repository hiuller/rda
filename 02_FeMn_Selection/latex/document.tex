%%% LaTeX Template: Article in MS Word style
%%%
%%% Source: http://www.howtotex.com/
%%% Feel free to distribute this template, but please keep the referal to HowToTeX.com.
%%% Date: January 2012

%%% ------------------------------------------------------------
%%% BEGIN PREAMBLE
%%% ------------------------------------------------------------

%%% document setup
% \documentclass[10pt,dvipdfm]{article} 
\documentclass[10pt]{article} 
\usepackage[super,square,sort,comma,numbers]{natbib}



%%% entradas em português
\usepackage[brazilian]{babel}
\usepackage[utf8]{inputenc}
\usepackage[T1]{fontenc}

%\usepackage{url}

%%% standard packages
\usepackage{amsmath,amsfonts,amsthm}
\usepackage{graphicx}
% \usepackage[dvipdfm]{graphicx}
\usepackage{multicol}

% http://tex.stackexchange.com/questions/75828/insert-figure-in-a-multicol-article
\usepackage{float}

% http://tex.stackexchange.com/questions/822/change-the-font-of-figure-captions
% change the font size for the captions (labels), could use: small, footnotesize and scriptsize.
\usepackage[font={footnotesize}]{caption}


%%% font
% \documentclass{beamer} 
\usepackage[scaled]{helvet} % With " scaled " option
\usepackage{eulervm}

%\usepackage[scaled]{uarial}
%\renewcommand*\familydefault{\sfdefault} %% Only if the base font of the document is to be sans serif
%\usepackage[T1]{fontenc}

% \usepackage{mbtimes}  % não compilou

% \usepackage{mathtime} % não compilou

\usepackage[varg]{txfonts} %parece Times new Roman

% \usepackage{fouriernc} %bunita

\usepackage{mathpazo} % moderna, parece de livro

% \renewcommand{\rmdefault}{phv} 		% Arial
% \renewcommand{\sfdefault}{phv} 		% Arial





\usepackage{indentfirst}
%%% indentation and linespacing
\setlength{\parindent}{25pt}	% indentation on new paragraph
\setlength{\parskip}{9pt}	% vertical spacing on new paragraph
\setlength{\lineskip}{1pt}	% vertical spacing between lines

%%% margins
\usepackage[top=1.55cm, bottom=2.29cm, left=1.6cm, right=1.47cm]{geometry}
\setlength{\columnsep}{1cm}	% spacing between columns
\setlength{\belowcaptionskip}{0pt}	% spacing below captions
\setlength{\abovecaptionskip}{5pt}	% spacong above captions

%%%% headings and sectioning
\usepackage{titlesec}
\titlespacing{\section}{0pt}{6pt}{3pt}

\pagestyle{empty}
\usepackage{sectsty}

\makeatletter
   \def\@seccntformat#1{\csname the#1\endcsname.\quad}
\makeatother
\subsectionfont{\sffamily \normalsize}
\subsubsectionfont{\sffamily \normalsize}
\sectionfont{\sffamily \large}

%% dummy text
\usepackage{lipsum}

% http://tex.stackexchange.com/questions/88890/how-to-get-the-references-section-to-be-numbered-as-if-it-were-created-via-sect
\usepackage[numbib]{tocbibind}


%%% ------------------------------------------------------------
%%% BEGIN DOCUMENT
%%% ------------------------------------------------------------
\begin{document}

%%% Title, author
%%% ------------------------------------------------------------	
\begin{center}
{\sffamily \begin{huge}\textbf{
		Análise dos fatores que afetam o rendimento da adição de ligas de manganês durante o vazamento utilizando técnicas de projetos de experimentos (DOE)
	}\end{huge}\\[10pt]
{Hiuller C. Araujo}
}\\

	\begin{abstract}
  As ligas de manganês são adicionadas ao aço durante o vazamento do convertedor e/ou nos processos subsequentes de refino secundário. Neste trabalho, foi investigado o efeito do teor de oxigênio no final de sopro, do índice de passagem de escória e do teor visado de manganês sobre o rendimento em manganês obtido durante o vazamento. O rendimento foi calculado com base no peso da carga metálica utilizando um rendimento metálico fixo de 91\%. A técnica empregada para análise dos dados foi a de Projetos de Experimentos (DOE), adaptada para utilização de dados que já estavam disponíveis. A análise de variância mostrou que as incertezas oriúndas da utilização de estimativas para o cálculo do rendimento em manganês resultou numa variação superior à do efeito das variáveis sob estudo. Para que seja viável estudar alternativas para controlar o rendimento de manganês no futuro será preciso desenvolver uma metodologia mais precisa para medir o rendimento de incorporação.
	\end{abstract}
\end{center}





%%% Content
%%% ------------------------------------------------------------
\begin{multicols}{2}

	%%% Title, author
%%% ------------------------------------------------------------	
\begin{center}
{\sffamily \begin{huge}\textbf{
	\begin{spacing}{1.0}
	Desenvolvimento de novos padrões de sopro para aumentar o carbono fim de sopro na Aciaria 2
	\end{spacing}
	}\end{huge}\\[10pt]
{Hiuller C. Araujo}
}\\
	\begin{abstract}
	\end{abstract}
	\rule{4in}{0.5pt}
\end{center}
%%% Content
%%% ------------------------------------------------------------
% this is a single column
%\begin{multicols}{2}


\pagestyle{plain}
\setcounter{page}{1}
\pagenumbering{arabic}
% \thispagestyle{empty}
\section{Introdução}

\lipsum		
	
	
%%%%%%%%%%%%%%%%%%%%%%%%%%%%%%%%%%%%%%%%%%%%%%%%%%%%%%
\bibliographystyle{IEEE/abntex2-num}
% \bibliographystyle{IEEE/IEEEtranN}
% \bibliographystyle{plainnat}
% enable it only if you want to display some
% this command renames the reference section: http://tex.stackexchange.com/questions/12597/renaming-the-bibliography-page-using-bibtex
\renewcommand{\bibname}{Referências}
\bibliography{references}

%\end{multicols}	

\bibliographystyle{IEEE/abntex2-num}
% \bibliographystyle{IEEE/IEEEtranN}
% \bibliographystyle{plainnat}
% enable it only if you want to display some
% this command renames the reference section: http://tex.stackexchange.com/questions/12597/renaming-the-bibliography-page-using-bibtex
\renewcommand{\bibname}{Referências}
\bibliography{references}

\end{multicols}	
\end{document} 